\RequirePackage{docswitch}
% \flag is set by the user, through the makefile:
%    make note
%    make apj
% etc.
\setjournal{\flag}

%\documentclass[\docopts]{\docclass}
\documentclass[skiphelvet,twocolumn]{aastex63}
%\documentclass[skiphelvet,preprint,twocolumn]{lsstdescnote}
% You could also define the document class directly
%\documentclass[]{emulateapj}

% Custom commands from LSST DESC, see texmf/styles/lsstdesc_macros.sty
\usepackage{lsstdesc_macros}
\usepackage{rotating}
\usepackage{graphicx}
%\usepackage{diagbox}
\usepackage{multirow}
\usepackage{adjustbox}
\usepackage{comment}
\usepackage{mathtools}
\usepackage{textcomp}
\usepackage[toc,page]{appendix}

%\usepackage{pdflscape}
\graphicspath{{./}{./figures/}}
\bibliographystyle{apj}

% Add your own macros here:
\newcommand{\snrb}{\mbox{$SNR^b$}}
\newcommand{\snrbmin}{\mbox{$SNR^b_{min}$}}
\newcommand{\snrg}{\mbox{$SNR^g$}}
\newcommand{\snrr}{\mbox{$SNR^r$}}
\newcommand{\snri}{\mbox{$SNR^i$}}
\newcommand{\snrz}{\mbox{$SNR^z$}}
\newcommand{\snry}{\mbox{$SNR^y$}}
\newcommand{\z}{{$z$}}
\newcommand{\bu}{{$u$}}
\newcommand{\bg}{{$g$}}
\newcommand{\br}{{$r$}}
\newcommand{\bi}{{$i$}}
\newcommand{\bz}{{$z$}}
\newcommand{\by}{{$y$}}
\newcommand{\salt}{SALT2}
\newcommand{\xnorm}{$x_0$}
\newcommand{\strech}{$x_1$}
\newcommand{\snstrech}{\mbox{$x_1$}}
\newcommand{\col}{$c$}
\newcommand{\daymax}{$T_0$}
\newcommand{\sigc}{\mbox{$\sigma_c$}}
\newcommand{\sigmu}{\mbox{$\sigma_\mu$}}
\newcommand{\zlim}{\mbox{$z_{lim}$}}
\newcommand{\zlimfaint}{\mbox{$z_{lim,faint}^{SN}$}}
\newcommand{\cosmos}{{COSMOS}}
\newcommand{\elais}{{ELAIS-S1}}
\newcommand{\xmm}{{XMM-LSS}}
\newcommand{\cdfs}{{CDF-S}}
\newcommand{\adfa}{{ADF-A}}
\newcommand{\adfb}{{ADF-B}}
\newcommand{\adfs}{{Euclid/Roman}}
\newcommand{\euclid}{{Euclid}}
\newcommand{\romanspace}{{Roman Space Telescope}}
\newcommand{\wfirst}{{\sc WFIRST}}
\newcommand{\sne}{{SNe~Ia}}
\newcommand{\sn}{{SN}}
\newcommand{\degsq}{{deg$^2$}}
\newcommand{\nsn}{{$N_{SN}^{z\leq z_{lim}}$}}
\newcommand{\nsncomp}{{$N_{SN}^{z\leq z_{complete}}$}}
\newcommand{\sumnsncomp}{{$\sum\limits_{Npixels} N_{SN}^{z\leq z_{complete}}$}}
\newcommand{\zcomp}{\mbox{$z_{complete}$}}
\newcommand{\zcompb}{\mbox{$z_{complete}^{0.95}$}}
\newcommand{\snfaint}{\mbox{$(\snstrech,\sncolor)=(-2.0,0.2)$}}
\newcommand{\snx}{\mbox{$x_0$}}
\newcommand{\sncolor}{\mbox{$c$}}
\newcommand{\redshift}{\mbox{$z$}}
\newcommand{\per}{$\%$}
\newcommand{\seq}{$\sim$}
\newcommand{\nvisits}{$N_{visits}$}
\newcommand{\nvisitsb}{\mbox{$N_{visits}^b$}}
\newcommand{\sumnvisitsb}{\mbox{$\sum\limits_{b}N_{visits}^b$}}
\newcommand{\nvisitsbmin}{\mbox{$N_{visits,min}^b$}}
\newcommand{\nvisitsy}{$N_{visits}^y$}
\newcommand{\nvisitsall}{$N_{visits}^g,N_{visits}^r,N_{visits}^i,N_{visits}^z,N_{visits}^y$}
\newcommand{\ddfscen}[1]{RDD\_#1}
\newcommand{\osfamily}[1]{{\it #1}}
\newcommand{\doffset}{tdo}

% ======================================================================

\begin{document}

\title{LSST Deep Drilling program and Supernovae}

% Author list file generated with: mkauthlist 1.2.4 
% mkauthlist -j tex -f -c LSST Dark Energy Science Collaboration --cntrb contributions.tex authors.csv authors.tex 

\author{Ph.~Gris}
\affiliation{Laboratoire de Physique de Clermont (LPC) , CNRS/IN2P3, F-63000 Clermont-Ferrand, France}
\author{N.~Regnault}
\affiliation{Laboratoire de Physique Nucléaire et des Hautes Energies, IN2P3/CNRS, France}


\collaboration{(LSST Dark Energy Science Collaboration)}


\maketitlepre

\begin{abstract}
  
  The ambition of the Rubin Observatory Legacy Survey of Space and Time (LSST) Supernovae (\sn) program is to unveil the nature of Dark Energy by measuring cosmological parameters with a high degree of accuracy. Reducing systematic uncertainties on cosmological parameters requires to enrich the core sample of well-measured Type Ia supernovae \sne. The average quality of observed \sne~is estimated from the signal-to-noise ratio of the photometric light curves which is primarly driven by survey parameters such as the cadence and the number of visits per band. An optimal observing strategy is thus critical in the success of the Supernovae science program and the LSST 
%Collecting a large sample of well-measured type Ia supernovae is a prerequisite to reduce systematic uncertainties on cosmological parameters. The average quality of the photometric light curves is primarily driven by observing strategy %The main Wide Fast Deep (WFD) survey will provide an unprecedented number of well-sampled \sne~up to redshift completeness of $\sim$0.2-0.3.
Deep Drilling Field (DDF) mini-survey is crucial for collecting a sample of high-redshift (up to $z\simeq$1) of well-measured supernovae which improves cosmological constraints from \sne. The goal of this paper is to quantify the impact of the DDF strategy parameters (number of fields, cadence, number of season of observation, number of visits per band, budget) on the redshift completeness and number of \sne. Detailed analysis of recent LSST simulations show that the depth of the DD survey (limited to $z\sim0.65$) could be explained by an insufficient signal-to-noise ratio (per band) or equivalently by an inappropriate number of visits per band. We propose a method that provides a guidance in the estimation of the number of visits (per band) requested to reach higher redshifts. The results of this method are used to precise the contours of DDF scenarios that would lead to the observation of a large sample of well-measured \sne~up to higher redshifts. We have taken synergistic dataset into account (Euclid/Roman and spectroscopy for follow-up and host-galaxy redshift) to build a realistic DD survey characterised by a low cadence (one day), a rolling strategy (each field observed at least two seasons), and two sets of fields: ultra-deep ($z\gtrsim0.9$) and deep ($z\gtrsim0.5$) fields. Further studies (i.e. accurate joint simulations with concerned endeavours) are requested to finalize the settings of this Deep Rolling (DR) strategy.

\end{abstract}

% Keywords are ignored in the LSST DESC Note style:
\dockeys{Cosmology; Supernovae; Astronomy software; Observing Strategy}

%\maketitlepost

% ----------------------------------------------------------------------
% 

\section{Introduction}
\label{sec:intro}
Type Ia supernovae(\sne) are transient astronomical events resulting from a powerful and luminous explosion of a white dwarf. They are identified by their brightness evolution, with a luminosity peak about 15 days after explosion, and a slow decrease lasting up to few months. \sne~can be used as standard candles to determine cosmological distances. They are included in a Hubble diagram, one is the most statistically efficient approach to constraint the Dark Energy equation of state (\citealt{Betoule_2014,Scolnic_2018}).
\par
The stated ambition of the Rubin Observatory Legacy Survey of Space and Time (LSST) Supernovae program is to maximize the sample size of well-measured type Ia supernovae while reducing systematic uncertainties on cosmological parameters. Increasing the statistics in the Hubble diagram requires (1) advances in the measurement of the distances (i.e.  a control at the per-mil level of the photometry and survey flux calibration, and a progress in standardization technique) (2) a better control of the astrophysical environment and its potential impacts on the SN light curves and distances (local host properties, absorption) (3) a better control of the SN diversity (SN~Ia sub-populations, population drift with redshift) (4) a precise determination of the survey selection function (SN identification, residual contamination by non-SN~Ia's as a function of redshift). Having access to a {\it large sample} of {\it well-measured} \sne~is a prerequisite for a successful completion of this list of improvements.
\par
The ten-year Rubin Observatory Legacy Survey of Space and Time (LSST) will image billions of objects in six colors. 80 to 90\per~of the observing time will be dedicated to Wide Fast Deep(WFD) primary survey, which will cover half of the sky ($\sim$ 18000 \degsq) at a ''universal'' cadence. The remaining observing time will be shared among other programs (mini-surveys) including intensive scanning of a set of Deep Drilling (DD) fields. It is expected that about 10\per~(100\per) of \sne~observed in the WFD (DD) survey will be identified from spectral features. Accurate supernova parameters will be estimated from well-measured light curves characterized by a sampling of few days and high Signal-to-Noise Ratio per band (\snrb).  %As a consequence all the studies presented in this paper rely on the supernova light curves only.
Obtaining high quality light curves is therefore a key design point of the SN survey:  the average quality of the light curves depends primarily on the observing strategy.
\par
In a recent paper (\cite{lochner2021impact}), the Dark Energy Science Collaboration (DESC) has presented an analysis of the WFD survey of observing strategies simulated by the LSST project\footnote{These strategies are available from https://community.lsst.org/t/community-survey-strategy-highlights/4760.}. The conclusion is that an unprecedented number of high quality \sne~will be observed in the WFD survey (between 120k and 170k) up to redshifts $z\sim 0.3$. The DD Supernovae program is critical for obtaining a sample of high-redshift (up to $z\simeq$1) and well-measured supernovae so as to achieve Stage IV dark energy goals.

%which improves cosmological constraints from \sne. Achieving Stage IV dark energy goals will critically rely on the deep drilling fields of LSST.
\par
%This paper compiles a set of studies related to the DD program of LSST.
This paper deals with the interplay between the DD strategy and \sne. The main goal is to design a set of strategies that would optimize the size and depth of the well-measured \sne~sample collected by the survey. We perform a detailed study of the impact of the strategy parameters (number of fields to observe, number of seasons, season lengths, numer of visits per night and per field) on the SN sample quality to assess whether observing \sne~up to $z\simeq$1 is achievable given design constraints, including in particular the number of visits alloted to DDFs. This article is subdivided into 5 sections. The design constraints of the DD program and the metrics used to assess observing strategies are described in the first two parts. A detailed analysis of recent DD simulations proposed by LSST is presented in the third section. A method to optimize the DD program is proposed in a fourth part. The last section of the document is dedicated to the presentation of  DD realistic scenarios that would achieve the goal of observing high quality \sne~up to high redshifts.

\section{Deep Drilling survey design constraints}
\label{sec:design}
Survey parameters having an impact on the design of the DD mini-surveys are the number of fields to be observed, the cadence of observation, the number of season of observation, the season length, and, last but not least, the budget.
\par
Four distant extragalactic survey fields that LSST guarantees to observe as Deep Drilling Fields were selected in 2012\footnote{http://ls.st/bki}: \cosmos, \elais, \xmm, \cdfs~(Tab. \ref{tab:locddf}). More recently, the DESC collaboration has supported the LSST DDF coverage of the southern deep fields area to ensure contemporaneous observations with \euclid~(\citealt{laureijs2011euclid,Amendola_2013}) and \romanspace~(\citealt{spergel2015widefield}), at the begining and at mid-term of the LSST survey, respectively.
\begin{table}[!htbp]
  \caption{Location of the DD fields considered in this study. ADF-A and ADF-B are example of south fields in the \adfs~area simulated in LSST observing strategy.}\label{tab:locddf}
  \begin{center}
    \begin{tabular}{c|c|c}
      \hline
      \hline
      Field & Central RA & Central Dec\\ 
      Name & (J2000)  & (J2000)\\
      \hline
     \elais & 00:37:48 & −44:01:30 \\
     \xmm & 02:22:18 &  −04:49:00 \\
     \cdfs & 03:31:55 & −28:07:00 \\
     \cosmos &10:00:26 & +02:14:01 \\
     \hline 
     \adfa & 04:51:00& −52:55:00 \\
     \adfb & 04:35:00 & −54:40:00 \\
      \hline
      \hline
      \end{tabular}
  \end{center}
\end{table}
\par
A regular cadence\footnote{The cadence is defined as the median of inter-night gaps.} of observation ($\sim$ 3 days max) is required to collect well-sampled light curves (LC). The number of large gaps ($>$10 days) between visits degrades the 
measurements of luminosity distances, and potentially result in rejecting large sets of light curves of poor quality.
\par
The number of exploding supernovae is proportional to the number of season of observation and to the season duration (\citealt{perrett}). Maximizing season length is particularly important in the DDFs because of time dilation. Season lengths of at least six months are required to maximize the size of the sample of well-measured \sne.
\par
It is expected that 5-15$\%$ of the total number of LSST visits will be alloted to the DD program and shared among science topics interested by DD observations (such as AGN, supernovae, ...).  This limited budget is related to the total number of visits per observing night by:
\begin{equation}\label{eq:ddbudget}
\begin{aligned}
&DD_{budget} = N_{visits}^{DD}/(N_{visits}^{DD}+ N_{visits}^{non-DD})\\
 &N_{visits}^{DD} = \sum_{i=1}^{N_{fields}} \sum_{ j=1}^{N_{season}^i} N_{visits,night}^{ij}\times seaslen^{ij}/cad^{ij} \\
 & N_{visits,night}^{ij} =  \sum_{b} N_{visits,night}^{ijb}   , b=g,r,i,z,y
 \end{aligned}
 \end{equation}

  %\begin{eqnarray}
  %DD_{budget} &=& N_{visits}^{DD}/N_{visits}^{tot} \\
  %N_{visits}^{DD} &=& \sum_{i=1}^{N_{fields}} \sum_{ j=1}^{N_{season}^i} N_{visits,night}^{ij}\times seaslen^{ij}\times 30/cad^{ij} \\
  %N_{visits,night}^{ij} &=& \sum_{b} N_{visits,night}^{ijb}   , b=g,r,i,z,y
  %\label{eq:ddbudget}
  %\end{eqnarray}
where $N_{fields}$ is the number of DD fields, $N_{season}$ the number of season of observations per field, $seaslen$ the season length (in days), $cad$ the cadence of observation, and $N_{visits, night}^{ij}$ the total number of visits per observing night, per field, and per season. The maximum number of visits per night and per field is fairly strongly dependent on the budget (the total number of visits is multiplied by 2.5 when increasing the budget from 6\% to 15\%) but also on the configuration of the survey that may be parametrized by $N_{fields}\times N_{seasons}^{field}$: the number of visits increases by a factor 5 if $N_{fields}\times N_{seasons}^{field}$ decreases from 50 to 10.

\begin{comment}
  We have estimated $N_{visits,night}$ from the following input: cadence of 1 day, season lengths of 6 months, 5 fields observed for 10 or 2 seasons, and a budget of 6, 10 and 15\%.  The conclusion (Tab. \ref{tab:ddbudget}) is that the maximum number of visits per night and per fields is quite dependent on the budget (the number of visits is multiplied by 2.5 when increasing the budget from 6\% to 15\%) but also on the configuration of the survey that may be parametrized by $N_{fields}\times N_{seasons}^{field}$: the number of visits increases by a factor 5 if $N_{fields}\times N_{seasons}^{field}$ decreases from 50 to 10. 

\begin{table}[!htbp]
  \caption{Total number of visits (per observing night) as a function of the DD budget and the cadence of observation for a configuration of 5 fields, for 10 and 2 seasons of observation, and for a cadence of 1 day.  }\label{tab:ddbudget}
  \begin{center}
    %\begin{tabular}{c|c|c|c}
    \begin{tabular}{c|c|c}
      \hline
      \hline
      %\diagbox[innerwidth=3.cm,innerleftsep=-1.cm,height=3\line]{budget}{cadence} & 1 & 2 & 3\\
      budget & $N_{visits, night}$ & $N_{visits, night}$\\
                    & 5 fields, 10 seasons & 5 fields, 2 seasons \\
      \hline
      6\% & 13 & 66 \\
      10\% & 22 & 111 \\
      15\% & 33 & 166 \\
      %6\% & 13/66 & 26/132 & 40/199 \\
      %10\% & 22/111 & 44/221 & 66/332 \\
      %15\% & 33/166 & 66/332 & 99/498 \\
      \hline
    \end{tabular}
  \end{center}
\end{table}
\end{comment}

\section{Metrics to assess observing strategies}
\label{sec:metrics}
The metrics used to assess observing strategies are estimated from full simulation of light curves. We have used the SALT2 model (\citealt{Guy_2007,Guy_2010}) where a \sne~is described by five parameters: \snx, the normalization of the SED sequence; \snstrech, the stretch; \sncolor, the color; \daymax, the day of maximum luninosity; and $z$, the redshift. A flat-$\Lambda$CDM model was used to estimate cosmological distances, with $H_0$ = 70 km s$^{-1}$, $\Omega_m$ = 0.3 and $\Omega_\Lambda$ = 0.7.
\par
In SALT2, model uncertainties of $g$ and $r$-band (rest-frame UV) light-curves fluxes are large (\citealt{Guy_2007}). $g$ and $r$ observations with relative error model larger than 10$\%$ have not been considered in this study. This requirement implies that the list of filters useful to measure photometric light-curves (observer-frame) is redshift-dependent: \bg\br\bi~for $z\lesssim$0.1,  \bg\br\bi\bz~for $0.1\lesssim z\lesssim$0.3-0.4, \br\bi\bz\by~for $0.4\lesssim z \lesssim 0.6-0.7$, and \bi\bz\by~for $z\gtrsim 0.7$.
\par
We rely on two metrics to assess observing strategies: the redshift limit \zlim, and the number of well-measured \sne, \nsn. A well-measured \sne~is defined by the following tight selection criteria: light curves points with $SNR\geq$ 1; at least four (ten) epochs before (after) maximum luminosity; at least one point with a phase lower (higher) than -10 (20); and \sigc$\leq$0.04 where \sigc~ is the error on the color parameter of the supernova (this corresponds to \sigmu$\leq$0.12 where \sigmu~is the distance modulus error). The redshift limit is defined as the maximum redshift of supernovae passing these selection criteria. The metrics are measured in HEALPixels of size 0.46 $deg^2$ over the region of the DDFs.
\par
The estimation of \zlim~is mainly driven by the selection on the error of the color, \sigc$\leq$0.04. \sigc~reflects the quality of the collected light curves, result of the cadence of observation (sampling) and of the flux uncertainty measurements (observing conditions). \sigc~estimation is strongly correlated to the Signal-to-Noise Ratio (SNR) per band, \snrb,  defined by:
\begin{equation}
  \begin{aligned}
    SNR^b &= \sqrt{\sum_{i=1}^{n^b}{\left(\frac{f_i^b}{\sigma_i^b}\right)^2}}
    \end{aligned}
  \label{eq:snrb}
\end{equation}
where $f^b$, and $\sigma^b$ are the fluxes and fluxes uncertainties. The summation runs over the number of light curve points. Requesting \sigc$\leq 0.04$ is equivalent to requiring a minimal SNR per band and the link between \zlim~and \snrb~may be written:
\begin{equation}
  \begin{aligned}
    \zlim &\Longleftarrow & \sigc \leq 0.04 & \Longleftarrow &\cap (\snrb \geq \snrbmin)
    \end{aligned}
 \label{eq:zlimsnr}
\end{equation}
The redshift of a complete sample, \zcompb, is estimated from the redshift limit distribution, \zlimfaint, of a simulated set of faint supernovae with \daymax~values spanning over the season duration of a group of observations. \zcompb~is defined as the 95th percentile of the \zlimfaint~cumulative distribution.


\section{Analysis of recent simulations}
\label{sec:analysis}
The LSST project has periodically released few sets of simulations during the past years. These simulations contain a large number of WFD strategies depending on parameters such as observed area, filter allocation, weather impact, scanning strategy. The diversity of DD scenarios is rather limited and we have choosen to analyze a set of observing strategies with representative DD surveys on the basis of the following criteria: number of visits (and filter allocation) per observing night, cadence of observation, dithering, and budget. The list of observing strategies is given in Tab. \ref{tab:os}.

%\begin{table}[!htbp] 
\caption{Survey parameters for the list of observing strategies analyzed in this paper. For the cadence and season length, the numbers correspond to ADFS1/ADFS2/CDFS/COSMOS/ELAIS/XMM-LSS fields, respectivelly. The numbers following the filter allocation (\nvisits~column) are the minimum and maximum mean fraction of visits (per field over seasons) for the corresponding filter distribution. Only filter combination with a contribution higher then 0.01 have been considered.}\label{tab:os} 
\begin{adjustbox}{width=1.1\linewidth,right} 
\begin{tabular}{c|c|c|c|c|c|c} 
  Observing & cadence & \nvisits & season length & area & DD budget & family\\ 
 Strategy & [days] & u/g/r/i/z/y & [days] & [deg2] &(\%)\\ 
\hline 
agnddf\_v1.5\_10yrs & 2.0/2.0/2.0/2.0/2.0/2.0 & -/1/1/3/5/4 [0.99-1.] & 164/165/235/189/171/177 & 112.9 & 3.4 & \osfamily{agn} \\ 
\hline 
baseline\_v1.5\_10yrs & 4.5/4.5/10.0/4.0/4.5/5.0 & -/10/20/20/26/20 [0.28-0.43] & 131/131/200/164/150/152 & 109.7 & 4.6 & \osfamily{baseline} \\
                                          &                                        & 8/10/20/20/-/20 [0.56-0.71] & & &  \\
\hline 
daily\_ddf\_v1.5\_10yrs & 2.0/2.0/2.0/2.0/2.0/2.0 & -/1/1/2/2/2 [0.60-0.61] & 161/161/236/188/171/178 & 113.5 & 5.5 & \osfamily{daily}\\
                                               &                                       & 1/1/1/2/-/2 [0.38-0.39] & & & \\
\hline 
ddf\_heavy\_v1.6\_10yrs & 2.0/2.0/2.0/2.0/2.0/2.0 & -/10/20/20/26/20 [0.26-0.39] & 116/116/201/167/152/150 & 110.6 & 13.4 & \osfamily{baseline}\\
                                                &                                       & 8/10/20/20/-/20 [0.60-0.72] & &  & \\
\hline 
&  & -/2/4/8/-/- [0.37-0.5] &  & & \\
descddf\_v1.5\_10yrs & 2.0/2.0/3.0/2.0/2.0/2.5 & -/-/-/-/25/4 [0.30-0.38] & 147/146/228/178/165/171 & 112.5 & 4.6 & \osfamily{desc}\\
 &  & -/-/-/-/-/4 [0.19-0.25] &  & & \\
\hline 
dm\_heavy\_v1.6\_10yrs & 7.5/6.0/14.0/8.5/8.0/7.0 & -/10/20/20/26/20 [0.31-0.45] & 119/119/195/142/139/138 & 188.6 & 4.6 & \osfamily{baseline}\\
                                                &                                        & 8/10/20/20/-/20 [0.54-0.68] &  &   &  \\
\hline 
ddf\_dither0.00\_v1.7\_10yrs & 4.0/4.0/6.0/2.0/3.0/3.0 & -/10/20/20/26/20 [0.17-0.43] & 121/123/204/165/153/159 & 69.2 & 4.6 & \osfamily{baseline}\\
                                                        &                                      & 16/10/20/20/-/20 [0.56-0.81] & 121/123/204/165/153/159 & 69.2 & 4.6 \\
\hline 
ddf\_dither0.05\_v1.7\_10yrs & 4.0/4.0/6.0/2.0/3.0/3.0 & -/10/20/20/26/20 [0.16-0.42] & 116/116/218/168/153/161 & 71.8 & 4.6 & \osfamily{baseline}\\
& & 16/10/20/20/-/20 [0.57-0.83] &&& \\ 
\hline
ddf\_dither0.10\_v1.7\_10yrs & 4.0/4.0/6.0/2.0/3.0/3.0 & -/10/20/20/26/20[0.19-0.43] & 120/120/220/165/150/165 & 74.7 & 4.6 & \osfamily{baseline}\\
& & 16/10/20/20/-/20 [0.57-0.81] &&& \\ 
\hline
ddf\_dither0.30\_v1.7\_10yrs & 4.0/4.0/6.5/3.0/3.0/3.0 & -/10/20/20/26/20 [0.21-0.45] & 118/118/201/167/146/146 & 83.5 & 4.6 & \osfamily{baseline}\\
& & 16/10/20/20/-/20 [0.54-0.78] &&& \\ 
\hline
ddf\_dither0.70\_v1.7\_10yrs & 4.5/4.5/9.0/4.0/4.0/4.25 & -/10/20/20/26/20 [0.19-0.43] & 123/137/201/163/146/146 & 104.5 & 4.6 & \osfamily{baseline}\\
& & 16/10/20/20/-/20 [0.57-0.79] &&& \\ 
\hline
ddf\_dither1.00\_v1.7\_10yrs & 4.0/4.0/14.0/5.5/5.0/5.0 & -/10/20/20/26/20 [0.23-0.43] & 113/118/198/153/143/143 & 124.4 & 4.6 & \osfamily{baseline}\\
& & 16/10/20/20/-/20 [0.56-0.77] &&& \\ 
\hline
ddf\_dither1.50\_v1.7\_10yrs & 4.5/4.5/16.5/8.5/6.75/6.0 & -/10/20/20/26/20 [0.20-0.42] & 121/121/196/145/135/139 & 159.3 & 4.6 & \osfamily{baseline}\\
& & 16/10/20/20/-/20 [0.57-0.79] &&& \\ 
\hline 
ddf\_dither2.00\_v1.7\_10yrs & 4.0/4.0/19.0/12.0/9.5/9.0 & -/10/20/20/26/20 [27-44] & 112/111/193/137/118/133 & 199.3 & 4.6 & \osfamily{baseline}\\
& & 16/10/20/20/-/20 [0.57-0.79] &&& \\ 
\end{tabular} 
\end{adjustbox} 
\end{table} 

%\begin{center} 
\resizebox{\textwidth}{!}{% 
\begin{tabular}{c|c|c|c|c|c|c|c|c|c} 
 Observing & DD budget(\%) & Field & cadence & Nvisits & Nnights & season length & area \\ 
 Strategy &  &  & min/med/max & g/r/i/z/y & & [days] & [deg2] \\ 
\hline 
& & ADFS1 & 2/4/34 & 10/20/20/26/20 & 5/11/13 & 11/121/148 & 15 \\ 
& & ADFS2 & 2/4/34 & 10/20/20/26/20 & 5/11/13 & 10/123/148 & 15 \\ 
ddf\_dither0.00\_v1.7\_10yrs& 4.6& CDFS & 3/6/15 & 10/20/20/26/20 & 10/21/26 & 62/204/228 & 9 \\ 
& & COSMOS & 2/2/10 & 10/20/20/26/20 & 18/22/24 & 142/165/177 & 10 \\ 
& & ELAIS & 2/3/17 & 10/20/20/26/20 & 5/22/24 & 68/153/168 & 8 \\ 
& & XMM-LSS & 2/3/13 & 10/20/20/26/20 & 7/22/25 & 58/159/172 & 9 \\ 
\hline 
& & ADFS1 & 2/4/32 & 10/20/20/26/20 & 5/11/15 & 28/116/149 & 15 \\ 
& & ADFS2 & 2/4/44 & 10/20/20/26/20 & 5/11/15 & 26/116/149 & 15 \\ 
ddf\_dither0.05\_v1.7\_10yrs& 4.6& CDFS & 2/6/27 & 10/20/20/26/20 & 5/21/25 & 52/218/228 & 9 \\ 
& & COSMOS & 2/2/30 & 10/20/20/26/20 & 5/22/24 & 78/168/177 & 10 \\ 
& & ELAIS & 2/3/35 & 10/20/20/26/20 & 5/22/24 & 49/153/168 & 9 \\ 
& & XMM-LSS & 2/3/29 & 10/20/20/26/20 & 5/23/24 & 54/161/169 & 10 \\ 
\hline 
& & ADFS1 & 1/4/29 & 10/20/20/26/20 & 5/11/13 & 8/120/145 & 15 \\ 
& & ADFS2 & 2/4/39 & 10/20/20/26/20 & 5/11/13 & 11/120/145 & 15 \\ 
ddf\_dither0.10\_v1.7\_10yrs& 4.6& CDFS & 2/6/45 & 10/20/20/26/20 & 5/21/25 & 56/220/228 & 10 \\ 
& & COSMOS & 2/2/29 & 10/20/20/26/20 & 5/21/24 & 46/165/177 & 10 \\ 
& & ELAIS & 2/3/39 & 10/20/20/26/20 & 5/22/24 & 50/150/168 & 11 \\ 
& & XMM-LSS & 2/3/37 & 10/20/20/26/20 & 5/22/25 & 32/165/172 & 10 \\ 
\hline 
& & ADFS1 & 2/4/32 & 10/20/20/26/20 & 5/10/13 & 28/118/148 & 15 \\ 
& & ADFS2 & 2/4/32 & 10/20/20/26/20 & 5/11/13 & 27/118/148 & 15 \\ 
ddf\_dither0.30\_v1.7\_10yrs& 4.6& CDFS & 2/6/50 & 10/20/20/26/20 & 5/20/25 & 49/201/230 & 13 \\ 
& & COSMOS & 2/3/47 & 10/20/20/26/20 & 5/21/24 & 33/167/177 & 12 \\ 
& & ELAIS & 2/3/38 & 10/20/20/26/20 & 5/21/25 & 28/146/168 & 13 \\ 
& & XMM-LSS & 2/3/44 & 10/20/20/26/20 & 5/21/25 & 37/146/174 & 13 \\ 
\hline 
& & ADFS1 & 2/4/39 & 10/20/20/26/20 & 5/11/13 & 28/123/157 & 15 \\ 
& & ADFS2 & 2/4/31 & 10/20/20/26/20 & 5/11/13 & 29/137/157 & 15 \\ 
ddf\_dither0.70\_v1.7\_10yrs& 4.6& CDFS & 2/9/63 & 10/20/20/26/20 & 5/14/25 & 29/201/228 & 18 \\ 
& & COSMOS & 2/4/42 & 10/20/20/26/20 & 5/17/24 & 26/163/177 & 17 \\ 
& & ELAIS & 2/4/42 & 10/20/20/26/20 & 5/17/25 & 28/146/168 & 18 \\ 
& & XMM-LSS & 2/4/40 & 10/20/20/26/20 & 5/15/24 & 34/146/169 & 18 \\ 
\hline 
& & ADFS1 & 2/4/31 & 10/20/20/26/20 & 5/10/13 & 24/113/147 & 15 \\ 
& & ADFS2 & 2/4/32 & 10/20/20/26/20 & 5/10/13 & 24/118/147 & 15 \\ 
ddf\_dither1.00\_v1.7\_10yrs& 4.6& CDFS & 2/14/55 & 10/20/20/26/20 & 5/11/25 & 49/198/230 & 23 \\ 
& & COSMOS & 2/5/47 & 10/20/20/26/20 & 5/13/24 & 32/153/177 & 23 \\ 
& & ELAIS & 2/5/44 & 10/20/20/26/20 & 5/12/25 & 13/143/177 & 23 \\ 
& & XMM-LSS & 2/5/47 & 10/20/20/26/20 & 5/12/25 & 39/143/172 & 23 \\ 
\hline 
& & ADFS1 & 2/4/39 & 10/20/20/26/20 & 5/10/13 & 27/121/145 & 15 \\ 
& & ADFS2 & 2/4/31 & 10/20/20/26/20 & 5/10/13 & 27/121/145 & 15 \\ 
ddf\_dither1.50\_v1.7\_10yrs& 4.6& CDFS & 2/16/59 & 10/20/20/26/20 & 5/10/25 & 28/196/230 & 31 \\ 
& & COSMOS & 2/8/49 & 10/20/20/26/20 & 5/10/24 & 30/145/174 & 32 \\ 
& & ELAIS & 2/6/42 & 10/20/20/26/20 & 5/10/24 & 27/135/168 & 32 \\ 
& & XMM-LSS & 2/6/45 & 10/20/20/26/20 & 5/10/25 & 26/139/176 & 32 \\ 
\hline 
& & ADFS1 & 2/4/39 & 10/20/20/26/20 & 5/10/13 & 23/112/149 & 15 \\ 
& & ADFS2 & 2/4/33 & 10/20/20/26/20 & 5/10/13 & 23/111/149 & 15 \\ 
ddf\_dither2.00\_v1.7\_10yrs& 4.6& CDFS & 2/19/58 & 10/20/20/26/20 & 5/9/20 & 29/193/229 & 41 \\ 
& & COSMOS & 2/12/46 & 10/20/20/26/20 & 5/8/20 & 29/137/173 & 42 \\ 
& & ELAIS & 2/9/52 & 10/20/20/26/20 & 5/9/22 & 21/118/168 & 42 \\ 
& & XMM-LSS & 2/9/50 & 10/20/20/26/20 & 5/9/22 & 22/133/174 & 41 \\ 
\end{tabular}} 
\end{sidewaystable} 
\end{center}
%\input{texfiles/dd_summary_2}
%\begin{center} 
\begin{sidewaystable}[htbp] 
\resizebox{\textwidth}{!}{% 
\begin{tabular}{c|c|c|c|c|c|c|c|c|c} 
 Observing & DD frac(\%) & Field & cadence & Nvisits & m5 & Nnights & season length & total area & effective area \\ 
 Strategy &  &  & min/med/max & g/r/i/z/y & g/r/i/z/y & & [days] & [deg2] & [deg2] \\ 
\hline 
& & ADFS1 & 2/4/31 & 10/20/20/26/20 & 25.8/25.8/25.4/24.7/23.9 & 5/10/13 & 24/113/147 & 15 & 12 \\ 
& & ADFS2 & 2/4/32 & 10/20/20/26/20 & 25.8/25.7/25.3/24.7/23.9 & 5/10/13 & 24/118/147 & 15 & 12 \\ 
ddf\_dither1.00\_v1.7\_10yrs& 4.6& CDFS & 2/14/55 & 10/20/20/26/20 & 25.6/25.6/25.1/24.6/23.7 & 5/11/25 & 49/198/230 & 23 & 18 \\ 
& & COSMOS & 2/5/47 & 10/20/20/26/20 & 25.6/25.6/25.2/24.7/23.8 & 5/13/24 & 32/153/177 & 23 & 17 \\ 
& & ELAIS & 2/5/44 & 10/20/20/26/20 & 25.7/25.6/25.2/24.7/23.8 & 5/12/25 & 13/143/177 & 23 & 18 \\ 
& & XMM-LSS & 2/5/47 & 10/20/20/26/20 & 25.6/25.6/25.1/24.7/23.7 & 5/12/25 & 39/143/172 & 23 & 19 \\ 
\hline 
& & ADFS1 & 2/4/39 & 10/20/20/26/20 & 25.8/25.8/25.3/24.7/23.9 & 5/10/13 & 27/121/145 & 15 & 13 \\ 
& & ADFS2 & 2/4/31 & 10/20/20/26/20 & 25.8/25.8/25.3/24.8/23.9 & 5/10/13 & 27/121/145 & 15 & 12 \\ 
ddf\_dither1.50\_v1.7\_10yrs& 4.6& CDFS & 2/16/59 & 10/20/20/26/20 & 25.6/25.6/25.1/24.6/23.7 & 5/10/25 & 28/196/230 & 31 & 22 \\ 
& & COSMOS & 2/8/49 & 10/20/20/26/20 & 25.6/25.6/25.1/24.7/23.7 & 5/10/24 & 30/145/174 & 32 & 23 \\ 
& & ELAIS & 2/6/42 & 10/20/20/26/20 & 25.7/25.7/25.2/24.8/23.8 & 5/10/24 & 27/135/168 & 32 & 22 \\ 
& & XMM-LSS & 2/6/45 & 10/20/20/26/20 & 25.6/25.5/25.1/24.7/23.7 & 5/10/25 & 26/139/176 & 32 & 22 \\ 
\hline 
& & ADFS1 & 2/4/39 & 10/20/20/26/20 & 25.8/25.8/25.3/24.7/23.9 & 5/10/13 & 23/112/149 & 15 & 13 \\ 
& & ADFS2 & 2/4/33 & 10/20/20/26/20 & 25.8/25.7/25.3/24.7/23.9 & 5/10/13 & 23/111/149 & 15 & 12 \\ 
ddf\_dither2.00\_v1.7\_10yrs& 4.6& CDFS & 2/19/58 & 10/20/20/26/20 & 25.6/25.6/25.1/24.6/23.6 & 5/9/20 & 29/193/229 & 41 & 25 \\ 
& & COSMOS & 2/12/46 & 10/20/20/26/20 & 25.6/25.6/25.2/24.7/23.7 & 5/8/20 & 29/137/173 & 42 & 26 \\ 
& & ELAIS & 2/9/52 & 10/20/20/26/20 & 25.7/25.7/25.2/24.7/23.8 & 5/9/22 & 21/118/168 & 42 & 27 \\ 
& & XMM-LSS & 2/9/50 & 10/20/20/26/20 & 25.6/25.6/25.1/24.7/23.7 & 5/9/22 & 22/133/174 & 41 & 26 \\ 
\end{tabular}} 
\end{sidewaystable} 
\end{center}
Four sets of observing strategies may be extracted from Tab. \ref{tab:os} according to the filter allocation per night, which is the parameter that has the most significant impact on the redshift limit value : \osfamily{baseline}~ (11 observing strategies), with two sets of filter distributions, \osfamily{agn}, \osfamily{daily}, and \osfamily{desc} family. Estimation of the pair metric (\nsncomp,\zcompb) (defined in Sec. \ref{sec:metrics}) for these families shows (Fig. \ref{fig:nsn_zlim_zoom}) that higher redshift limits are reached for the \osfamily{baseline-like} family. Most (10/11) of these observing strategies reach \zcompb$\sim$0.65. ddf\_heavy, the strategy with the largest budget, reaches \zcompb$\sim$0.72 and also collects the larger number of well-sample \sne. \osfamily{daily} and \osfamily{desc} are characterized by a lower depth but by a significant number of well-measured \sne.\par
The redshift completeness value is mainly driven by the number of visits per observing night and by the cadence. We would expect \zcompb$\sim$(0.75,0.52,0.48,0.65) for (\osfamily{baseline},\osfamily{agn},\osfamily{daily},\osfamily{desc}), respectivelly, for a regular cadence of 2 days and median observing conditions. The corresponding expected number of well-measured supernovae would be of (9000,3400,2700,6200) without dithering. A comparison of these numbers with Fig. \ref{fig:nsn_zlim_zoom} reveals that the values of the pair metric %(\nsncomp, \zcompb)
are the complex outcome of the combination of the following (ranked by order) parameters: number of visits per observing night$\times$cadence, season length, dithering, observing conditions. Gaps of more than $\sim$ 5-7 days in a 2 days median cadence have a harmful
impact on the size and depth of the SN sample. The main source of gaps for the DDF survey is due to telescope downtime (clouds, telescope maintenance) which leads to about 16-20$\%$ of nights without observation per season.  

\begin{figure*}[htbp]
\begin{center}
  \includegraphics[width=0.8\textwidth]{nsn_zlim_zoom.png}
 \caption{\nsncomp~vs \zcompb~for the observing strategies considered in this paper.}\label{fig:nsn_zlim_zoom}
\end{center}
\end{figure*}

\par
The translationnal dithering is expected to affect both the number of well-sampled supernovae and the redshift completeness for each of the HEALPixel of a field. With no dithering, \nsncomp~and \zcompb~distributions are uniform on the whole field area. The dithering tends increase the cadence of observation and to decrease both \nsncomp~and \zcompb~(per HEALPixel). The dithered pointings of the telescope are estimated from the central location of the fields (Tab. \ref{tab:locddf}) and lead to a distribution map of the metrics decreasing as the (HEALPixel center) distance to the central location increases. A more subtle effect is observed (Fig. \ref{fig:dither}) when all the pixel are considered to estimate the total number of supernovae and the 95th percentile redshift completeness. \zcompb~tends to decrease with an increase of the translationnal dither offset (\doffset), with a greater effect for low cadences. The total number of supernovae is the result of two opposite effects, a decrease of the cadence and an increase of the survey area, the later offsetting the former for low cadences and low \doffset~values.

\begin{figure}[htbp]
  \begin{center}
  \includegraphics[width=0.55\textwidth]{dither_ddf.png}
 \caption{Ratio of the number of supernovae $N_{SN}/N_{SN}^{nodither}$ (top) and \zcompb~difference (bottom) as a function of translationnal dither offset. The simulations labelled as 'Fakes' (dotted lines) correspond to regular cadences (1,2,3,5 days) with median observing conditions (5$\sigma$ depth single exposure: 24.13/23.84/23.45/22.74/22.10 for $g/r/i/y/z$ bands, respectivelly.}\label{fig:dither}
\end{center}
\end{figure}



\section{Optimization of the number of visits}
\label{sec:opti}

The analysis of recent simulations have shown (see Sec. \ref{sec:analysis}) that it was difficult, with the proposed cadences of observation, filter allocation and season lengths, to collect complete samples of \sne~with redshift higher than \zcompb\seq 0.55-0.6 for a DD budget of \seq 5\per.  %A high budget (\seq 13.5\per) is requested to reach \zcompb\seq 0.65.
%The redshift limits are mainly driven by \sumnvisitsb$\times$ cadence.
With optimal survey parameters (i.e. a regular cadence of observation of one day, no dithering, minimal fluctuation of observing conditions) the depth of the four families considered in Sec. \ref{sec:analysis} would be \zcompb $\sim$ (0.77,0.59,0.54,0.72) for (\osfamily{baseline},\osfamily{agn},\osfamily{daily},\osfamily{desc}), respectivelly. These results are well below the ambitious goal of \zcompb\seq1 because \snrb~values are too low to reach higher redshifts (Eq. \ref{eq:zlimsnr}). The signal-to-noise ratio per band is the complex result of the combination of the SN flux ($z$-dependent), the number of visits per, the cadence of observation, and observing conditions (5-$\sigma$ depth). It is thus not possible to estimate the 
observing strategy parameters required to reach higher redshifts from the results of Sec. \ref{sec:analysis}. This is why we present in this section a study to assess the relationship between the redshift limit and the number of visits per band and per observing night (for a defined cadence). The optimized number of visits per band required to reach higher redshifts estimated with this approach is a key parameter to build DD scenarios consistent with the list of constraints presented in Sec. \ref{sec:design}.
\par
As described in Eq. \ref{eq:ddbudget} the DD budget depends primarily on 5 parameters: the number of fields to observe, the season length (per field and per season), the number of seasons of observation, the cadence of observation (per field and per season), and the number of visits \nvisitsb~per filter and per observing night. \nvisitsb~affect \snrb~through the flux measurement uncertainties  $\sigma_i^b$. In the background-dominated regime one has $\sigma_i^b \simeq \sigma_5^b$ where $\sigma_5^b$ is equal by definition to
\begin{equation}\label{eq:opt2}
  \begin{aligned}
    \sigma_5^b &=  \frac{f_5^b}{5}
    \end{aligned}
\end{equation}
where $f_ 5^b$ is the five-sigma flux related to the five-sigma depth magnitude $m_5^b$ through:
\begin{equation}\label{eq:opt3}
  \begin{aligned}
    m_5^b &= -2.5 \log f_5^b+zp^b
    \end{aligned}
\end{equation}
where $zp^b$ is the zero point of the considered filter.  $m_5^b$ is related to \nvisitsb through:
\begin{equation}\label{eq:opt4}
  \begin{aligned}
    m_5^b - m_5^{b, single~visit} & =  1.25 \log(N_{visits}^b)
    \end{aligned}
\end{equation}
where $m_5^{b, single~visit}$ is the five-sigma depth corresponding to a single visit, a parameter depending on observing conditions. These equations \eqref{eq:opt2}-\eqref{eq:opt4} describe the relationship between \snrb~ and \nvisitsb. The request $\snrb~\geq~\snrbmin$ is equivalent to $\nvisitsb~\geq~\nvisitsbmin$ and equ. \eqref{eq:zlimsnr} may be written:
\begin{equation}
  \begin{aligned}
    \zlim &\Longleftarrow & \sigc \leq 0.04 & \Longleftarrow &\cap (\nvisitsb~\geq~\nvisitsbmin)
    \end{aligned}
 \label{eq:zlimnvisits}
\end{equation}
\\
The relations \eqref{eq:zlimsnr} and \eqref{eq:zlimnvisits} are not univocal. A lot of \snrb~(\nvisitsb) combinations lead in fact to the same result and constraints have to be applied to choose the optimal configuration. We have performed a systematic scan of the SNR parameter space (\snrg,\snrr,\snri,\snrz,\snry) and picked combinations fulfilling the above-mentioned selection criteria (Sec. \ref{sec:metrics}). The optimal solution is selected by minimizing the total number of visits per observing night and by requiring a maximum number of \by-band visits. This selection aims at reducing the (potentially severe) systematic effects affecting the \by~band measurements. The result is displayed on Fig. \ref{fig:nvisits_zlim} for a 1 day cadence. The number of visits strongly increases with the redshift completeness for \zcomp$\gtrsim 0.7$ where only three bands \bi\bz\by~can be used to construct \sne~light curves. 284 visits are required to reach \zcomp$\sim$0.95 for a one day cadence. The number of visits required to reach a given \zcomp~value increases linearly (as a first approximation) with the cadence. This tend to disfavour DD strategies with high cadences: a high number of visits per observing night (e.g. 7 hours of observations to reach \zcomp $\sim$ 0.95 for a cadence of 3 days) would jeopardize the uniformity of the WFD survey.

\begin{figure}[htbp]
  \includegraphics[width=0.5\textwidth,left]{nvisits_zlim.png}
 \caption{Number of visits as a function of the completeness redshift. 235 visits {\it with} the following filter allocation (\nvisitsall)=(2,4,89,121,18) are requested per observing night to reach \zcomp\seq 0.9 for a cadence of one day.}\label{fig:nvisits_zlim}
\end{figure}
\par
The optimized number of visits required to reach higher redshift completeness is the last piece of the puzzle to be included in the budget estimator (Eq. \ref{eq:ddbudget}). We have now the tools to design realistic DD scenarios.
%leading to the collection of a large-sample of well measured \sne~up to higher completeness redshifts.

\begin{comment}
We may then use \ref{eq:zliminvisits} and \ref{fig:nvisits_zlim} to estimate the redshift completeness corresponding to the number of visits of \ref{tab:ddbudget}. The conclusions of the result (Tab. \ref{tab:zlim}) are (a) it is very difficult to reach completeness redshifts higher than 0.6-0.7 if all fields are observed for ten years ; (b) the only way to explore higher  redshift domains is to reduce the number of seasons of observation.

\begin{table}[!htbp]
  \caption{Redshift completeness as a function of the DD budget and the cadence of observation for a configuration of 5 fields. The first/second number corresponds to 10/2 seasons of observation per field. The redshift completeness are independent on the cadence since the total SNR per band, \snrb,are identical.}\label{tab:zlim}
  \begin{center}
    \begin{tabular}{c|c|c|c}
      \hline
      \hline
      \diagbox[innerwidth=3.cm,innerleftsep=-1.cm,height=3\line]{budget}{cadence} & 1 & 2 & 3\\
      \hline
      6\% &\multicolumn{3}{c}{0.62/0.74} \\
      10\% & \multicolumn{3}{c}{0.66/0.79} \\
      15\% & \multicolumn{3}{c}{0.68/0.83}\\
      \hline
    \end{tabular}
  \end{center}
\end{table}
\end{comment}

\section{Deep Rolling surveys to probe high \zcomp~domains}
\label{sec:scenario}
\paragraph{Season length and field location}
An astronomical target is said to be observable if it is visible (i.e. for LSST with altitude 20$^o\leq$ alt$\leq$86.5$^o$ and airmass$\leq$1.5) for some time. The nightly period of visibility depends on the location of a field w.r.t. LSST. The season length may be estimated from a list of night during which a field is observable. It depends on the time of visibility that is the total exposure time of a field. The estimation of the season length as a function of the total number of visits (Fig. \ref{fig:seasonlength_nvisits}) suggests a decrease from 275-200 to 150-100 days when \nvisits~increase from 1 to 400.  Combining information of sec. \ref{sec:opti} and Fig. \ref{fig:seasonlength_nvisits} lead to the conclusion that low cadences are favored to reach higher \zcomp~values while maximizing the season length.
%This is due to the fact that the minimal \snrb~to reach \zcomp~is independent on the cadence. The corresponding requested number of visits increases with the cadence. This leads to a decrease of the season length.

\begin{figure}[htbp]
\begin{center}
  \includegraphics[width=0.5\textwidth]{seasonlength_nvisits.png}
 \caption{Maximal season length as a function of the number of visits. Fields are observable if the following requirements are met: altitude 20$^o\leq$ alt$\leq$86.5$^o$ and airmass$\leq$1.5.}\label{fig:seasonlength_nvisits}
\end{center}
\end{figure}

\paragraph{Deep Rolling strategy}

A Deep Drilling program may be defined by the following parameters: number of fields to observe, number of season of observation (per field), season length (per field and per season),  number of visits per filter per observing night, DD budget, number of supernovae, and redshift completeness.  Once the field parameters configuration (fields to observe, number of seasons, season length) is set, one of the three parameters (\zcomp, budget, \nvisits) may be fixed to estimate the two others using Eq. \ref{eq:ddbudget} and the results of Fig.~\ref{fig:nvisits_zlim}. 
\begin{comment}
A GUI (Fig. \ref{fig:budget_gui}) was designed from the results of sec. \ref{sec:opti} to design Deep Drilling scenarios. Once the field parametersconfiguration (fields to observe, number of seasons, season length) is set, one of the three parameters (\zcomp, budget, \nvisits) may be fixed to estimate the two others. 

\begin{figure}[htbp]
\begin{center}
  \includegraphics[width=0.95\textwidth]{budget_GUI.png}
 \caption{A GUI to define Deep Drilling programs. The field parameters (fields to observe, number of seasons, season length) are defined in the bottom table. The bottom plot displays the number of visits as a function of \zcomp. The budget as a function of \zcomp~is represented on the top plot. One of the three parameters (\zcomp, budget, \nvisits) may be fixed to estimate the two others. The expected total number of supernovae is also estimated. We have chosen \zcomp$\sim$0.8 as an illustraion.}\label{fig:budget_gui}
\end{center}
\end{figure}
\end{comment}

Observing 5 fields for ten years up to \zcomp $\sim$ 0.9 would certainly give access to a large sample of well-measured \sne~(around 11k) but also to an unrealistic scenario (DD budget: 87\%). The only way to reach higher \zcomp~while remaining within a reasonable budgetary envelope is to reduce the number of seasons of observation. This is the purpose of the Deep Rolling (DR) strategy characterized by a limited number of seasons of observation per field (at least 2) and a large number of visits per observing night (more than 200 for higher \zcomp).
\par
%Building a consistent DDR strategy is not only a matter of adjusting LSST survey settings to stay within a reasonable budgetary envelope.
%Synergistic
Spectroscopic datasets from cosmological endeavors overlapping with LSST in area and timing are essential for the success of the supernovae program. They provides enormous added benefits through (a) the follow-up of the full sample of well-measured supernovae (photometric classification), and (b) the measurement of host-galaxy redshifts with high accuracy. Two of the spectroscopic resources contemporaneous with LSST, Primary Focus Spectrograph (PFS \cite{Tamura_2016}) and 4MOST(\cite{4MOST}), may provide guidance in the choice of the depth of the DR survey. The PFS spectroscopic follow-up survey is designed to observe two of LSST DDFs accessible from the Subaru telescope: \cosmos~and \xmm. About 4000 spectra (half of live supernovae and half of host galaxy redshifts) up to $z\sim0.8$ will be collected. The 4MOST Time-Domain Extragalactic Survey (TiDES \cite{TiDES}) is dedicated to the spectroscopic follow-up of extragalactic optical transients and variable sources selected from e.g. LSST. The goal is to collect spectra for up to 30 000 live transients to $z\sim0.5$ and to mesure up to 50 000 host galaxy redshifts up to $z\sim1$. Two sets of LSST fields may then be defined to fully benefit from the synergy with PFS and 4MOST: ultra-deep (\cosmos, \xmm) and deep (\adfs, \cdfs, \elais) fields.

\begin{figure*}[htbp]
\begin{center}
  \includegraphics[width=0.8\textwidth]{budget_zcomp.png}
  \caption{DD budget (top) and total number of \sne~(bottom) as a function of the redshift completeness of deep fields for a set of scenarios where the redshift completeness of ultra-deep fields (\cosmos~and \xmm) is set to 0.9. Subscripts correspond to the number of seasons of observation and superscripts to the redshift limit. Highest \zcomp~values are reached for a minimal strategy composed of two seasons of observations of \cosmos~and \xmm~with a depth of 0.9 plus 4 seasons of observation of the \adfs~ field (1 pointing). Coloured areas correspond to a variation of the number of $y$-band visits (20$\leq N_{visits}^y \leq$ 80).}\label{fig:budget_zcomp}
\end{center}
\end{figure*}

\par
On the basis of the above, it is possible to sketch the outlines of a realistic large scale high-$z$ DD survey: (a) a low cadence of observation (one day), (b) a rolling strategy (with a minimal of two seasons of observation per field), and (c) two sets of fields, ultra-deep (\cosmos,\xmm, with \zcomp$\gtrsim$0.8) and deep (\adfs,\cdfs,\elais, with \zcomp$\gtrsim$0.5) fields. The budget and expected number of \sne~for a set of scenario is given on Fig. \ref{fig:budget_zcomp}. A budget of 8-9$\%$ is requested to observe ultra-deep fields up to $z\sim0.9$ and deep fields up to $z\sim0.7$ and to collect 1100 to 2100 \sne. The parameters of the DR strategy (number of fields to observe, redshift completeness for ultra-deep and deep fields, number of seasons per field,budget) have to be tuned on the basis of studies with the synergistic surveys (PSF, TiDES, Euclid, Roman) using accurate joint simulations.
\par
The sequence of observations (field/night) of the DR survey must fulfill a couple of constraints. LSST observation of \adfs~has to be contemporaneous with \euclid~(years 2 and 3) and with \romanspace~(years 5 and 6). Observing multiple fields per night is not optimal if the number of visits per field is high. It may jeopardize the uniformity of the WFD survey (if the total number of DD visits is too high) and have a negative impact on the regularity of the cadence (if a choice has to be made among the DDFs). Overlap of field observations should be minimized. This means that the DR survey should be deterministic with a timely sequence defined in advance. An example of the progress of a DR survey is given on Fig. \ref{fig:timelysequence} for a configuration of 5 fields and a depth of 0.9 and 0.7 for ultra-deep and deep fields, respectivelly. 

\begin{figure}[htbp]
\begin{center}
  \includegraphics[width=0.5\textwidth,height=0.3\textwidth]{timely_sequence_235visits.png}
  \caption{ Cumulative sum of the number of nights (per field and per season) as a function of the time since surveystart (assumed to be late 2023).  The following sequence of observations is considered: \cosmos, \adfs~(x2), \xmm, \cosmos, \adfs~(x2), \elais~(x2), \xmm ,  with  a  maximum  season  length  of  180  days for the deep fields,  a  cadence  of  one day, and ensuring only one field is observed per night.  The fields are required to be observable (airmass $\leq$ 1.5 and 20\textdegree $\leq$ altitude $\leq$ 86.5\textdegree) for at least 2 hours (235 visits) for the ultra-deep fields and 15' (29 visits) for the deep fields.  The overlap, defined as the fraction of nights with more than one field observed during a night, is $\sim7\%$}\label{fig:timelysequence}
\end{center}
\end{figure}

\begin{comment}
\begin{table}[!htbp]
  \caption{Set of scenarios to reach higher \zcomp.}\label{tab:rolling_scenarios}
  \begin{center}
    \begin{tabular}{c|c|c|c|c|c|c|c}
      \hline
      \hline
      Scenario & \zcomp & \nsncomp & budget & \nvisits & fields & seasons & season length\\
                       &                 &                     &               & \bg/\br/\bi/\bz/\by &        & & [month] \\
      \hline
      \multirow{5}{*}{\ddfscen{a}} & \multirow{5}{*}{0.8} & \multirow{5}{*}{2270} & \multirow{5}{*}{11.4\per} & & COSMOS & 1,2 & 5.8 \\
         &       &          &                  &          & CDFS         & 3,4 & 6.0 \\
         &       &          &                  &      127               & ELAIS        & 7,8 & 6.0 \\
         &       &          &                  &    2/2/45/58/19                  &XMM-LSS  & 9,10 & 6.0 \\
         &       &          &                  &                     &ADFS          & 1,2,5,6 & 6.0 \\
      \hline
      \multirow{5}{*}{\ddfscen{b}} & \multirow{5}{*}{0.84} & \multirow{5}{*}{2500} & \multirow{5}{*}{15.0\per} & & COSMOS & 1,2 & 5.4\\
         &       &          &                  &          & CDFS         & 3,4 & 6.0 \\
         &       &          &                  &      169               & ELAIS        & 7,8 & 6.0 \\
         &       &          &                  &    2/2/63/85/18                  &XMM-LSS  & 9,10 & 5.8\\
         &       &          &                  &                     &ADFS          & 1,2,5,6 & 6.0 \\
      \hline
      \multirow{3}{*}{\ddfscen{c}} & \multirow{3}{*}{0.9} & \multirow{3}{*}{1860} & \multirow{3}{*}{14.3\per} & & COSMOS & 1,2 & 4.7 \\
         &       &          &                  &     250     & CDFS         & 3,4 & 6.0 \\
         &       &          &                  &    2/2/98/130/18                 &ADFS          & 1,2,5,6 & 6.0 \\
      \hline
      \end{tabular}
  \end{center}
\end{table}
\end{comment}

% ----------------------------------------------------------------------

% ----------------------------------------------------------------------

%\section{Lessons from recent simulations}
%\label{sec:simu}

% ----------------------------------------------------------------------

%\section{Discussion}
%\label{sec:discussion}



% ----------------------------------------------------------------------

\section{Conclusions}
\label{sec:conclusion}
In this paper we have presented a set of studies assessing the impact of the LSST Deep-Drilling mini-survey on the size and depth of a sample of well-measured \sne. A comprehensive analysis of the recent LSST simulations has shown that it was difficult to collect a sample of higher resfhift \sne~without exceeding a reasonable budget allocation and that the proposed number of visits (per observing night) had to be adjusted. Reaching higher redshift completeness requires to increase the signal-to-noise ratio of the photometric light curves, while considering band-flux distribution ($z$-dependent), cadence, and observing conditions. We have proposed a method providing the relationship between the optimized number of visits {\it per band}  and the redshift completeness. We used this result to design a set of realistic strategies that would meet the initial requirements: collecting a large sample of well-measured \sne~up to higher redshift while respecting survey design constraints (cadence, budget, number of fields, season length). Synergistic datasets have been taken into account to lead to a Deep Rolling strategy with the following contours: (a) low cadence (one day), (b) rolling strategy (at least two seasons of obervation per field), (c) two sets of fields: ultra-deep (\cosmos, \xmm - \zcomp~$\gtrsim$~0.8) and deep (\adfs, \cdfs, \elais - \zcomp~$\gtrsim$~0.5).
\par
The minimal scenario proposed in this paper consists of three fields, two ultra-deep (\cosmos, \xmm, for two seasons) and one deep (\adfs, for four seasons in synergy with \euclid~and \romanspace). The redshift completeness will depend on the DD alloted budget, which is conditionned to the WFD survey time required to meet SRD requirements. It is not clear whether this information will be known prior to the start of the survey. This is why the proposed scenario is flexible and may be adapted according to (potentially changing) circumstances. The DD budget will provide guidance for the depth of the \sne~sample once the number of fields to observe is chosen. Accurate joint studies between LSST and external facilities providing contemporaneous observations (\euclid, \romanspace, Subaru, 4MOST) have to be performed to optimize the depth and size of the \sne~sample collected with the DR survey.



% ----------------------------------------------------------------------

\subsection*{Acknowledgments}

%%% Here is where you should add your specific acknowledgments, remembering that some standard thanks will be added via the \code{desc-tex/ack/*.tex} and \code{contributions.tex} files.

%This paper has undergone internal review in the LSST Dark Energy Science Collaboration. % REQUIRED if true

Author contributions are listed below. \\
Ph.~Gris: conceptualization,software,analysis,writing \\
N.~Regnault: conceptualization,writing \\
 % Standard papers only: author contribution statements. For examples, see http://blogs.nature.com/nautilus/2007/11/post_12.html

% This work used TBD kindly provided by Not-A-DESC Member and benefitted from comments by Another Non-DESC person.

% Standard papers only: A.B.C. acknowledges support from grant 1234 from ...

\input{desc-tex/ack/standard} % also available: key standard_short

% This work used some telescope which is operated/funded by some agency or consortium or foundation ...

% We acknowledge the use of An-External-Tool-like-NED-or-ADS.

%{\it Facilities:} \facility{LSST}

\appendix
%\begin{appendix}

\section{Observing Strategies analyzed in this paper}
\label{appendix:tab}
\begin{table}[!htbp] 
\caption{Survey parameters for the list of observing strategies analyzed in this paper. For the cadence and season length, the numbers correspond to ADFS1/ADFS2/CDFS/COSMOS/ELAIS/XMM-LSS fields, respectivelly. The numbers following the filter allocation (\nvisits~column) are the minimum and maximum mean fraction of visits (per field over seasons) for the corresponding filter distribution. Only filter combination with a contribution higher then 0.01 have been considered.}\label{tab:os} 
\begin{adjustbox}{width=1.1\linewidth,right} 
\begin{tabular}{c|c|c|c|c|c|c} 
  Observing & cadence & \nvisits & season length & area & DD budget & family\\ 
 Strategy & [days] & u/g/r/i/z/y & [days] & [deg2] &(\%)\\ 
\hline 
agnddf\_v1.5\_10yrs & 2.0/2.0/2.0/2.0/2.0/2.0 & -/1/1/3/5/4 [0.99-1.] & 164/165/235/189/171/177 & 112.9 & 3.4 & \osfamily{agn} \\ 
\hline 
baseline\_v1.5\_10yrs & 4.5/4.5/10.0/4.0/4.5/5.0 & -/10/20/20/26/20 [0.28-0.43] & 131/131/200/164/150/152 & 109.7 & 4.6 & \osfamily{baseline} \\
                                          &                                        & 8/10/20/20/-/20 [0.56-0.71] & & &  \\
\hline 
daily\_ddf\_v1.5\_10yrs & 2.0/2.0/2.0/2.0/2.0/2.0 & -/1/1/2/2/2 [0.60-0.61] & 161/161/236/188/171/178 & 113.5 & 5.5 & \osfamily{daily}\\
                                               &                                       & 1/1/1/2/-/2 [0.38-0.39] & & & \\
\hline 
ddf\_heavy\_v1.6\_10yrs & 2.0/2.0/2.0/2.0/2.0/2.0 & -/10/20/20/26/20 [0.26-0.39] & 116/116/201/167/152/150 & 110.6 & 13.4 & \osfamily{baseline}\\
                                                &                                       & 8/10/20/20/-/20 [0.60-0.72] & &  & \\
\hline 
&  & -/2/4/8/-/- [0.37-0.5] &  & & \\
descddf\_v1.5\_10yrs & 2.0/2.0/3.0/2.0/2.0/2.5 & -/-/-/-/25/4 [0.30-0.38] & 147/146/228/178/165/171 & 112.5 & 4.6 & \osfamily{desc}\\
 &  & -/-/-/-/-/4 [0.19-0.25] &  & & \\
\hline 
dm\_heavy\_v1.6\_10yrs & 7.5/6.0/14.0/8.5/8.0/7.0 & -/10/20/20/26/20 [0.31-0.45] & 119/119/195/142/139/138 & 188.6 & 4.6 & \osfamily{baseline}\\
                                                &                                        & 8/10/20/20/-/20 [0.54-0.68] &  &   &  \\
\hline 
ddf\_dither0.00\_v1.7\_10yrs & 4.0/4.0/6.0/2.0/3.0/3.0 & -/10/20/20/26/20 [0.17-0.43] & 121/123/204/165/153/159 & 69.2 & 4.6 & \osfamily{baseline}\\
                                                        &                                      & 16/10/20/20/-/20 [0.56-0.81] & 121/123/204/165/153/159 & 69.2 & 4.6 \\
\hline 
ddf\_dither0.05\_v1.7\_10yrs & 4.0/4.0/6.0/2.0/3.0/3.0 & -/10/20/20/26/20 [0.16-0.42] & 116/116/218/168/153/161 & 71.8 & 4.6 & \osfamily{baseline}\\
& & 16/10/20/20/-/20 [0.57-0.83] &&& \\ 
\hline
ddf\_dither0.10\_v1.7\_10yrs & 4.0/4.0/6.0/2.0/3.0/3.0 & -/10/20/20/26/20[0.19-0.43] & 120/120/220/165/150/165 & 74.7 & 4.6 & \osfamily{baseline}\\
& & 16/10/20/20/-/20 [0.57-0.81] &&& \\ 
\hline
ddf\_dither0.30\_v1.7\_10yrs & 4.0/4.0/6.5/3.0/3.0/3.0 & -/10/20/20/26/20 [0.21-0.45] & 118/118/201/167/146/146 & 83.5 & 4.6 & \osfamily{baseline}\\
& & 16/10/20/20/-/20 [0.54-0.78] &&& \\ 
\hline
ddf\_dither0.70\_v1.7\_10yrs & 4.5/4.5/9.0/4.0/4.0/4.25 & -/10/20/20/26/20 [0.19-0.43] & 123/137/201/163/146/146 & 104.5 & 4.6 & \osfamily{baseline}\\
& & 16/10/20/20/-/20 [0.57-0.79] &&& \\ 
\hline
ddf\_dither1.00\_v1.7\_10yrs & 4.0/4.0/14.0/5.5/5.0/5.0 & -/10/20/20/26/20 [0.23-0.43] & 113/118/198/153/143/143 & 124.4 & 4.6 & \osfamily{baseline}\\
& & 16/10/20/20/-/20 [0.56-0.77] &&& \\ 
\hline
ddf\_dither1.50\_v1.7\_10yrs & 4.5/4.5/16.5/8.5/6.75/6.0 & -/10/20/20/26/20 [0.20-0.42] & 121/121/196/145/135/139 & 159.3 & 4.6 & \osfamily{baseline}\\
& & 16/10/20/20/-/20 [0.57-0.79] &&& \\ 
\hline 
ddf\_dither2.00\_v1.7\_10yrs & 4.0/4.0/19.0/12.0/9.5/9.0 & -/10/20/20/26/20 [27-44] & 112/111/193/137/118/133 & 199.3 & 4.6 & \osfamily{baseline}\\
& & 16/10/20/20/-/20 [0.57-0.79] &&& \\ 
\end{tabular} 
\end{adjustbox} 
\end{table} 


%\end{appendices}



% Include both collaboration papers and external citations:
\bibliography{main,lsstdesc}

\end{document}

% ======================================================================
